\documentclass{article}
\overfullrule=1mm

\usepackage{hyperref}
\usepackage{url}
\usepackage{stmaryrd}
\usepackage{amsmath}

\newcommand{\DeclareRel}[1]{\texttt{(declare-rel #1 )}}
\newcommand{\DeclareVar}[1]{\texttt{(declare-var #1 )}}
\newcommand{\Xlate}[1]{\llbracket #1 \rrbracket}
\DeclareMathOperator{\VarVersion}{V}
\DeclareMathOperator{\Preserve}{Pres}

\begin{document}


\hypertarget{abstract}{%
\section{Abstract}\label{abstract}}

Relational formats describe which relations may be established by which
other relations. Transition systems describe which states may be reached
by which other states. We describe a safety-property-preserving
translation between a relational program description and a transition
system description. We identify a state for each element of a relation;
transitions through the system corresponding to relation derivations.

\hypertarget{relational-chc-description---what-relations-elements-can-you-establish}{%
\section{Relational CHC description - what relations elements can you
establish}\label{relational-chc-description---what-relations-elements-can-you-establish}}

A relational CHC description is a set of rules and a query. The rules
are horn clauses which tell you how you can establish certain relations.
A relation may be established outright or conditionally.

Here is an example program.

\begin{verbatim}
entry:
i = 0
bb0: do { i += 3; }
while (i < 5)
bb1: assert(i < 7)
either goes to bbsafe or bbunsafe
\end{verbatim}

Below is one way to express this program as Horn clauses.
The labels are translated into relations.
Each relation is (potentially) defined over
program variables.
In this example, relations bb0 and bb1 use
the values for i, so they take the current value of i as a parameter.

\begin{verbatim}
(declare-rel entry ())
(declare-rel bb0 (Int))
(declare-rel bb1 (Int))
(declare-rel bbsafe ())
(declare-rel bbunsafe ())
\end{verbatim}

We use i and x as (universally-quantified) variables inside the rules.
Essentially, these variables are temporaries.

\begin{verbatim}
(declare-var i Int)
(declare-var x Int)
\end{verbatim}

The rules correspond to steps in the execution of the program. We can
always get to entry, so we have a single rule for that. Each (rule
(=\textgreater{} X Y)) has a head Y and a body X. The body is the
condition upon which the head Y is established.

\begin{verbatim}
(rule entry)
(rule (=> (and entry (= i 0))
          (bb0 i)))
(rule (=> (and (bb0 i) (< i 5) (= x (+ i 3)))
          (bb0 x)))
(rule (=> (and (bb0 i) (not (< i 5)))
          (bb1 i)))
(rule (=> (and (bb1 i) (< i 7))
          bbsafe))
(rule (=> (and (bb1 i) (not (< i 7)))
          bbunsafe))
(query bbunsafe)
\end{verbatim}

For example, in these rules, we can establish the relation
\texttt{(bb0\ 0)} as long as we're at \texttt{entry} and \texttt{i=0}.
Note that \texttt{i} is universally quantified, so \texttt{i=0} is
really defining what \texttt{i} we're talking about.

In order to establish the query (\texttt{bbunsafe}), we would reason
backward to see what allows us to establish it (the last rule).
Recursively, we would reason backwards again to establish that rule's
body, etc.

\hypertarget{constructs-in-the-chc-language}{%
\subsection{Constructs in the CHC
language}\label{constructs-in-the-chc-language}}

\begin{verbatim}
(declare-var [var] [sort]) Declare a variable that is universally quantified in Horn clauses.
(declare-rel [relation-name] ([sorts])) Declare relation signature. Relations are uninterpreted.
(rule [universal-horn-formula]) Assert a rule or a fact as a universally quantified Horn formula.
(query [relation-name]) Pose a query. Is the relation non-empty.
(set-predicate-representation [function-name] [symbol]+) Specify the representation of a predicate.
\end{verbatim}

From here: \url{https://rise4fun.com/z3/tutorialcontent/fixedpoints}

%%%%%%%%%%%%%%%%%%%%%%%%%%%%%%%%%%%%%%%%%%%%%%%%%%%%%%%%%%%%%%%%%%%%%%
\hypertarget{transition-system-description---what-states-can-you-reach}{%
\section{Transition system description - what states can you
reach}\label{transition-system-description---what-states-can-you-reach}}

A transition system describes how to reach states in the states space.
Since we want to relate Horn systems to transition systems, we have to show what it means to
establish a relation.

The state space of our transition system will be composed of individual relation elements.
If the relation $(1,2)\in R$ is entailed by our Horn clauses, a corresponding state $x=1$, $y=2$ will be reachable in the transition system.
For a relation of arity $k$, we associate $k+1$ state variables.
One variable is Boolean, an indicator variable meaning, if true, ``the current state specifies an element of the relation.''
The other $k$ variables specify which element of the relation it is.
Every Horn \texttt{rule} corresponds to a transition, allowing more elements to be related.
\texttt{query} corresponds to the safety property.

%%%%%%%%%%%%%%%%%%%%%%%%%%%%%%%%%%%%%%%%%%%%%%%%%%%%%%%%%%%%%%%%%%%%%%
\hypertarget{translation-rules}{%
\subsection{Translation rules}\label{translation-rules}}

Recall that a transition system is defined over $(X, Y, X')$ using
formulas $I$, $T$, $P$. Initially the sets $X$, $Y$, $X'$ are empty and $I$, $T$, $P$,
are true.

\paragraph{The state space}
The following two CHC directives affect the state space of our transition system encoding.
\begin{center}
(declare-var $x$ $s$) -- Declare a variable that is universally quantified in Horn clauses.
\end{center}

The translation of this adds a variable $x$ of sort $s$ to $Y$.

\begin{center}
(declare-rel $R$ ($s_0 \ldots s_k$))  -- Declare relation signature. Relations are uninterpreted.
\end{center}

First, add a Boolean state variable $R$ to $X$. Next, for each formal
parameter position {$i$} of $R$, associate to it a unique
state variable $R_i$ of sort $s_i$. 
The correspondence to the Horn system is that a relation $(a,b,c,\ldots)\in R$ is derivable if and only if the state $(R \wedge R_0 = a \wedge R_1 = b \wedge R_2 = c \wedge \ldots)$ is reachable.

\paragraph{The transition relation}

\begin{center}
(rule $f$) -- Assert a rule or a fact as a universally quantified Horn formula.
\end{center}

A rule is translated to a conjunction over current and next state variables.
The
conjunction then becomes a a disjunct of $T$. A rule is separated into
a \emph{head} and optional \emph{body}. A relation occurrence's
translation depends on whether we are translating the body (which gives
conditions) or the head (which gives a result).
Let's define a helper, $\VarVersion(x)$, that returns the state variable $x$ if we're translating the body and the state variable $x'$ if we're translating the head.

Head or body is translated like so:
\begin{align}
  \Xlate{(R~x_0~x_1~\ldots~x_k)} &\rightarrow  \VarVersion(R) \wedge \VarVersion(R_0)=\Xlate{x_0} \wedge \ldots \wedge \VarVersion(R_k)=\Xlate{x_k} & \\
  \Xlate{(\text{and}~x~y)} &\rightarrow \Xlate{x} \wedge \Xlate{y} & \\
  \Xlate{(\text{not}~x)} &\rightarrow \neg\Xlate{x} & \\
  \Xlate{x} &\rightarrow \VarVersion(x) & \qquad{} \text{if $x \in X$} \\
  \Xlate{a} &\rightarrow a & 
\end{align}
The last case simply sends constants and primary input to their syntacticaly-identical counterparts.
Finally, the transition rule is formed by preserved the variable not updated in the translation of the rule.

Let's assume we're given three relations, R, Q, and Z.
Q is 0-arity and R and Z are 1-arity.
Then the following rule
\begin{center}
  \texttt{(rule (=> (and (R x) (= a (+ x 3))) (Z a)))}
\end{center}
has the following translation:
\begin{equation}
  R \wedge (R_0 = x) \wedge  (a = (x+3)) \wedge Z' \wedge Z_0'=a \wedge Q' = Q
\end{equation}

\begin{verbatim}
(query [relation-name]) Pose a query. Is the relation non-empty.
(set-predicate-representation [function-name] [symbol]+) Specify the representation of a predicate.
\end{verbatim}
 
When the transition system steps into a state where the
indicator variable \texttt{R} is true and its argument variables satisfy
\texttt{R1\ =\ a1}, \texttt{R2\ =\ a2}, \ldots{}, \texttt{Rn\ =\ an},
then the relation \texttt{(R\ a1\ ..\ an)} is established.

%\subsection{Example}

\begin{figure}
\begin{verbatim}
(declare-rel entry ())
(declare-rel bb0 (Int))
(declare-rel bb1 (Int))
(declare-rel bbsafe ())
(declare-rel bbunsafe ())
(declare-var i Int)
(declare-var x Int)
(rule entry)
(rule (=> (and entry (= i 0))
          (bb0 i)))
(rule (=> (and (bb0 i) (< i 5) (= x (+ i 3)))
          (bb0 x)))
(rule (=> (and (bb0 i) (not (< i 5)))
          (bb1 i)))
(rule (=> (and (bb1 i) (< i 7))
          bbsafe))
(rule (=> (and (bb1 i) (not (< i 7)))
          bbunsafe))
(query bbunsafe)
\end{verbatim}

% XXX use \mathrm in a newcommand to make the things look better
State variables $X = \{ entry, bb0, bb0_{0}, bb1, bb1_{0}, bbsafe, bbunsafe \}$

Primary inputs $Y = \{ i, x \}$

Initial state $I = \{ \neg entry, \neg bb0, \neg bb1, \neg bbsafe, \neg bbunsafe \}$

\begin{align}
  entry' \wedge \Preserve \\
entry \wedge i=1 \wedge bb0' \wedge bb0_{0}'=i \wedge \Preserve \\
bb0 \wedge bb0_{0} =i \wedge i<5 \wedge x=(i+3)) \wedge bb0' \wedge bb0_{0}'=x \wedge \Preserve \\
bb0 \wedge bb0_{0}=i \wedge \neg (i<5) \wedge bb1' \wedge bb1_{0}'=i \wedge \Preserve \\
bb1 \wedge bb1_{0}=i \wedge i<7 \wedge bbsafe' \wedge \Preserve \\
bb1 \wedge bb1_{0}=i \wedge \neg (i<7) \wedge bbunsafe' \wedge \Preserve
\end{align}

Property: $\neg bbunsafe$

\caption{The Horn clauses given above are reproduced here along with their translation. Each occurrences of $\Preserve$ refers to a formula that preserves the values of the state variables not modified in the corresponding equation.}
\end{figure}


\end{document}
